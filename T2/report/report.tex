\documentclass[conference]{IEEEtran}
\IEEEoverridecommandlockouts

\usepackage{cite}
\usepackage[portuguese,brazil]{babel}
\usepackage{amsmath, amssymb, amsfonts}
\usepackage[utf8]{inputenc}
\usepackage{algorithmic}
\usepackage{graphicx}
\usepackage{textcomp}
\usepackage{subfig}
\usepackage{diagbox}
\def\BibTeX{{\rm B\kern-.05em{\sc i\kern-.025em b}\kern-.08em
    T\kern-.1667em\lower.7ex\hbox{E}\kern-.125emX}}
\begin{document}

\title{MO444 - Machine Learning - Relatório da Atividade \#2 Grupo 12}

\author{\IEEEauthorblockN{Bárbara Caroline Benato}
\IEEEauthorblockA{
RA 192865\\
barbarabenato@gmail.com}
\and
\IEEEauthorblockN{Breno Leite}
\IEEEauthorblockA{
RA 192863\\
brenolleite@gmail.com}}

\maketitle

\section{Introdução}

A segunda atividade da disciplina de \textit{Machine Learning} (MO444) tem como objetivo explorar técnicas de classificação de padrões multiclasse, a fim de se encontrar um sistema de detecção de objetos que faça o reconhecimento e classificação das imagens presentes na base de dados \textit{Cifar-10}. Assim, a solução apropriada deve ser encontrada evitando o supertreinamento dos dados. 

O objetivo deste relatório é apresentar os experimentos que foram desenvolvidos com intuito de encontrar o melhor modelo para classificar objetos presentes na base, utilizando a Regressão Logística e Rede Neural.

O mesmo é dividido em Seções. Na Seção \ref{sec:meto}, alguns dados interessantes sobre a base de dados são mostrados, bem como a metodologia empregada e as atividades desenvolvidas. A Seção \ref{sec:exp} apresenta os experimentos e sua discussão. Por fim, a Seção \ref{sec:conc} discute os principais aprendizados e os objetivos que foram atingidos com o desenvolvimento do trabalho.

\section{Materiais e Métodos} \label{sec:meto}

Os materiais, como a base de dados e pacote utilizados, e a metodologia empregados no presente trabalho são descritos a seguir.

\subsection{Base de dados \textit{Cifar-10}} \label{sec:base}

A base de dados \emph{Cifar-10} é utilizada para a validação dos estudos e objetivos propostos para o presente trabalho. A base de dados apresenta imagens de diferentes objetos, bem como o rótulo de cada objeto. O rótulo das imagens de objetos estão dividos em 10 classes, como, por exemplo, avião, carro, ave, gato, veado, cachorro, sapo, cavalo, barco e caminhão. A base de dados apresenta $60.000$ imagens coloridas de tamanho $32 \times 32$ \textit{pixels}, que são  divididas em $50.000$ imagens de treinamento e $10.000$ imagens para teste.

\subsection{Pacote \textit{Scikit-Learn}} \label{sec:pac}

Os modelos desenvolvidos para a Regressão Logística foram implementados utilizando funções do pacote \emph{Scikit-Learn}, em linguagem de programação \emph{Python}. A função \emph{LogisticRegression} foi empregada a fim de se obter uma solução base de Regressão Logística, bem como para Regressão Logística Multinomial.

A configuração dos parâmetros da função \emph{LogisticRegression} foi dada da seguinte forma:
\begin{itemize}
	\item \textit{penalty}: Uma penalidade que utiliza uma constante de regularização \textit{l2} é utilizada para obter uma solução esparsa.
	\item \textit{dual}: Falso, uma formulação primária é utilizada devido ao número de amostras ser maior que o número de características.
	\item \textit{tol}: Valor de tolerância utilizado como critério de parada de $0.0001$.
	\item \textit{C}: Valor responsável por controlar a força da regularização adicionada atribuído como $1$.
	\item \textit{fit\_intercept}: Verdadeiro, ou seja, a intersecção com o eixo y é calculada.
	\item \textit{intercept\_scaling}: Falso, pois é incompatível com o algoritmo de otimização escolhido.
	\item \textit{class\_weight}: Nenhum tipo de balanceamento, uma vez que o conjunto de dados é balanceado.
	\item \textit{random\_state}: Nenhum tipo de randomização é adicionada, pois é incompatível com o algoritmo de otimização escolhido.
	\item \textit{solver}: Algoritmo de otimização escolhido definido como \textit{lbfgs}, apropriado para o problema multiclasse.
	\item \textit{max\_iter}: Número máximo de iterações definido como $50$.
	\item \textit{multi\_class}: Parâmetro que define se o problema deve ser encarado como \textit{one vs rest} ou \textit{multinomial}. A documentação do pacote define o problema \textit{one vs all} com mesmo parâmetro para o problema \textit{one vs rest}. Ambos os parâmetros são abordados no presente trabalho. 
	\item \textit{warm\_start}: Falso, não se utiliza coeficientes obtidos anteriormente.
\end{itemize}


\subsection{Biblioteca \textit{Keras}} \label{sec:bib}

Já para os modelos desenvolvidos considerando Rede Neural, uma biblioteca de interface de desenvolvimento em alto nível de redes neurais foi utilizada\footnote{https://keras.io/}. Tal biblioteca é escrita em \emph{Python}. Optou-se por utilizar como base do \textit{Keras}, a biblioteca \textit{Theano}, uma biblioteca que executa expressões matemáticas envolvendo vetores de muitas dimensões eficientemente.

A principal função utilizada para as camadas da Rede Neural é denominada \textit{Dense}. Tal função é capaz de representar redes neurais multicamadas totalmente conectada. A configuração de parâmetros é dada como segue:
\begin{itemize}
	\item \textit{units}: Dimensão da saída de cada camada definida como $3800$.
	\item \textit{activation}: Funções de ativação empregadas nos experimentos: linear, tangente hiperbólica, sigmoidal, relu e softmax. 
	\item \textit{use\_bias}: Verdadeiro, o vetor de bias é utilizado.
	\item \textit{kernel\_initializer}: Inicialização das matrizes de pesos utiliza inicialização uniforme Glorot, definida como padrão.
	\item \textit{bias\_initializer}: Bias inicializados com $0$.
	\item \textit{kernel\_regularizer}: Nenhuma função de regularização aplicada `as matrizes de peso.
	\item \textit{bias\_regularizer}: Nenhuma função de regularização aplicada ao vetor de bias.
	\item \textit{activity\_regularizer}:Nenhuma função de regularização aplicada `a saída da camada.
	\item \textit{kernel\_constraint}: Nenhuma constante de regularização aplicada `as matrizes de peso.
	\item \textit{bias\_constraint}: Nenhuma constante de regularização aplicada ao vetor de bias.
\end{itemize}

\subsection{Metodologia} \label{sec:met}

Um processo gradual foi adotado para solucionar o problema proposto neste trabalho. Primeiramente uma regressão logística foi aplicada, utilizando o método \emph{one-vs-all}. Em sequência um modelo de regressão multinomial foi implementado, esses modelos foram utilizados como \emph{baseline} para o trabalho.

Com intuito de melhorar os resultados, redes neurais foram utilizadas. Diversos experimentos foram executados para encontrar a melhor configuração da rede. Os experimentos deste trabalho são divididos em treino e teste, na parte de treino um processo de 5-\emph{fold} foi utilizado, onde os dados de teste nunca são utilizados. Após escolher os melhores modelos estes são testados utilizando da base de teste para verificar o resultado da nossa escolha.

\section{Experimentos e Discussões}

Esta seção tem como objetivo mostrar os resultados obtidos pelos modelos desenvolvidos neste trabalho, assim como explicar os diversos experimentos realizados para a escolha do modelo final.

\subsection{}



\begin{thebibliography}{00}
%\bibitem{b1} Christopher M. Bishop. ``Pattern Recognition and Machine Learning''. Springer-Verlag New York, Inc., Secaucus, NJ, USA, 2006.
\bibitem{b1} Aurélien Géron ``Hands-On Machine Learning with Scikit-Learn and TensorFlow
Concepts, Tools, and Techniques to Build Intelligent Systems". O'Reilly Media, March 2017.
\end{thebibliography}


\end{document}
